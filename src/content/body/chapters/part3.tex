\chapter{Finally}

And finaly the final part !
Let's talk about footnotes\footnote{This is my footnote of footnote!} and annexes\footnote{This is my footnote for anexe!} !

I am doing a ref to annex \ref{appendix:toto} !

\section{Doing some lists !}

Here is one \\
\begin{itemize}
  \item One entry in the list
  \item Another entry in the list
\end{itemize} 


\begin{enumerate}
  \item The labels consists of sequential numbers.
  \item The numbers starts at 1 with every call to the enumerate environment.
\end{enumerate}

\begin{enumerate}
   \item First level item
   \item First level item
   \begin{enumerate}
     \item Second level item
     \item Second level item
     \begin{enumerate}
       \item Third level item
       \item Third level item
       \begin{enumerate}
         \item Fourth level item
         \item Fourth level item
       \end{enumerate}
     \end{enumerate}
   \end{enumerate}
 \end{enumerate}
 
 \begin{itemize}
   \item  First Level
   \begin{itemize}
     \item  Second Level
     \begin{itemize}
       \item  Third Level
       \begin{itemize}
         \item  Fourth Level
       \end{itemize}
     \end{itemize}
   \end{itemize}
 \end{itemize}
 
 \section{For glossary}
 gob
 \section{First Section}

The \Gls{latex} typesetting markup language is specially suitable for documents that include \gls{maths}. \Glspl{formula} are rendered properly an easily once one gets used to the commands.


\clearpage

\section{Second Section}

\vspace{5mm}

Given a set of numbers, there are elementary methods to compute its \acrlong{gcd}, which is abbreviated \acrshort{gcd}. This process is similar to that used for the \acrfull{lcm}.
